\documentclass[]{article}

% Imported Packages
%------------------------------------------------------------------------------
\usepackage{amssymb}
\usepackage{amstext}
\usepackage{amsthm}
\usepackage{amsmath}
\usepackage{enumerate}
\usepackage{fancyhdr}
\usepackage[margin=1in]{geometry}
\usepackage{graphicx}
\graphicspath{ {images/} }
\usepackage{extarrows}
\usepackage{setspace}
\usepackage{blindtext}
\usepackage{scrextend}
\addtokomafont{labelinglabel}{\sffamily}
%------------------------------------------------------------------------------

% Header and Footer
%------------------------------------------------------------------------------
\pagestyle{plain}  
\renewcommand\headrulewidth{0.4pt}                                      
\renewcommand\footrulewidth{0.4pt}                                    
%------------------------------------------------------------------------------

% Title Details
%------------------------------------------------------------------------------
\title{Deliverable \#2 -- High-Level Architectural Design Document}
\author{SE 3A04: Software Design II -- Large System Design}
\author{Chen, Arthur \and Campbell, Christopher \and Endrizzi, Johnny \\ 
\and Coovert, Mitchell \and Gill, Surinder \and Dhadda, Terin}
\date{March 7, 2016}                              
%------------------------------------------------------------------------------

% Document
%------------------------------------------------------------------------------
\begin{document}



% Table of Contents
%------------------------------------------------------------------------------
\maketitle	
\newpage
\tableofcontents
\listoffigures
\listoftables
\newpage
%------------------------------------------------------------------------------

% Introduction
%------------------------------------------------------------------------------
\section{Introduction}
\label{sec:introduction}
The following section provides a brief overview of the entire document.

\subsection{Purpose}
\label{sub:purpose}
The purpose of this document is to lay out the high level architectural design of the "BEER'D" application. It will first give a description of the system and a general overview of what it is for, how it is expected to be used, and why it is being developed. It also contains information about the variety of use cases for the application, an analysis class diagram, a breakdown of the intended architecture design, and finally a class responsibility collaboration breakdown. This document is intended primarily for the developers of the application, the professor, and the teaching assistants.

\subsection{System Description}
\label{sub:system_description}
The "BEER'D" system is a mobile application that aims to solve the question: "What beer is this?" This application is primarily being developed as a project for the third year Software Architecture class (course code SE 3A04) taught at McMaster University. A team of 6 students will design, develop, and create the application.\\
\\
The "BEER'D" application will take specific inputs from a user. Based on these inputs, varying "experts" will attempt to analyse and come up with their best prediction (based on data provided by publicly available API's) as to which beer the inputs may be identifying. The application will return and display a list of possible answers in a forum. Within this forum, users will also be able to share their answers on popular social media networks or find local stores which sell the beers referred to in the answers - based on their current location in an map.

\subsection{Overview}
\label{sub:overview}
The rest of the document is split up into four main sections:
\begin{enumerate}[-]
	\item The first section, Use Case Diagram, will contain each use case associated with the application. 
	\item The second section, Analysis Class Diagram, will contain the analysis class diagram for the application based upon the use case diagram.
	\item The third section, Architectural Design, will provide an overview of the overall architectural design for the application. It will first identify and provide reasoning for the chosen software architecture. Then, it explain the division of the system into subsystems and describe each subsystem.
	\item The fourth and final section, Class Responsibility Collaboration, will contain the "CRC Cards" of the application.
\end{enumerate}
%------------------------------------------------------------------------------

% Use Case Diagram
%------------------------------------------------------------------------------
\newpage
\section{Use Case Diagram}
\label{sec:use_case_diagram}
The following section provides a use case diagram for the application. \\

\begin{figure}[!ht]
%\includegraphics[scale=0.45]{use_case_diagram}
\caption{Use Case Diagram for the BEER'D Application}
\end{figure}

\begin{labeling}{Update Beer Database}
\item [Home Page] The Home Page use case will generally be the one encountered the most, due to the nature of mobile applications. When a user starts the application, they will be directed to the home page, where they will begin interacting with the application.
\item [Request General Information] When a customer is curious and wishes to obtain general information about a beer, such as the different types, tastes, colors, etc. they can open this page which will display such information.
\item [Input Information] The Input Information use case is included in the Home Page use case. The user will be asked to select several different inputs for the experts to analyse. 
\item [Review Previous Searches] When the user wishes to review their previous searches (i.e. the results to their inputs from searches they had in the past) they will be able to view them.
\item [Delete Search] The user deletes a a search in their search history.
\item [Re-Search] The user searches again from a specific search in their history, using the same input data saved in the search. This use case will use the Access Information Controller to perform the search.
\item [Access Information Controller] This use case is an abstract use case. It is used by many other use cases. It uses the Input Information use case. It coordinates the use cases of taking input, consulting experts to analyse the input, accessing the database, and displaying the result to the forum. 
\item [Consult Experts] When the user has input their information for a desired search, the Access Information Controller will use this use case to analyse the inputs. This is where the experts will predict which beer the user's inputs are describing. It will use up to 3 experts to come up with the prediction. It will then return this information back to the controller.
\item [Expert 1/2/3 Analysis] These use cases will analyse the user's input that is specific to them, respectively. It will then come up with their best prediction(s) based on the input.
\item [Forum] When the search has been completed and the application is ready to display it's predictions, it will display them in this use case. The Forum use case uses the results from the Access Information Controller use case. 
\item [Share on Social Media] When the user wants to share their results on social media, they log in (or sync their social media accounts, encrypted by the application) and post them.
\item [Open Nearby Locations] When the application displays it's predicted results, it will display a map based on the user's current location and mark nearby beer retailers which sell the beers listed in the results.
\item [Edit Expert] This use case includes one of the primary functional requirements. The developer may deem it necessary to swap, add, or remove an existing expert. This use case extends Consult Experts since any modifications to the experts must be considered when consulting experts. In other words, the application should be using the must recently existing experts.
\item [Update Beer Database] When a new beer comes to the market, the developer may need to update the database to include information about this beer for the experts to be able to analyse and include in their predictions. This extends the Access Information Controller because any additions (or modifications in general) must be accessible to the controller. 
\end{labeling}
%------------------------------------------------------------------------------

\newpage
\section{Analysis Class Diagram}
\label{sec:analysis_class_diagram}
% Begin Section
This section should provide an analysis class diagram for your application.
% End Section


\section{Architectural Design}
\label{sec:architectural_design}
% Begin Section
This section should provide an overview of the overall architectural design of your application. You overall architecture should show the division of the system into subsystems with high cohesion and low coupling.

\subsection{System Architecture}
\label{sub:system_architecture}
% Begin SubSection
\begin{enumerate}[a)]
	\item Identify and explain the overall architecture of your system
	\item Be sure to clearly state the name of the architecture
	\item Provide the reasoning and justification of the choice
	\item Provide a structural architecture diagram showing the relationship among the subsystems (if appropriate)
\end{enumerate}
% End SubSection

\subsection{Subsystems}
\label{sub:subsystems}
% Begin SubSection
\begin{enumerate}[a)]
	\item Provide a brief description of each subsystem. Be sure to document its purpose and relationship to other subsystems.
\end{enumerate}
% End SubSection

% End Section
	
\section{Class Responsibility Collaboration (CRC) Cards}
\label{sec:class_responsibility_collaboration_crc_cards}
% Begin Section

	\begin{table}[ht]
		\centering
		\begin{tabular}{|p{5cm}|p{5cm}|}
		\hline 
		 \multicolumn{2}{|l|}{\textbf{Class Name: Experts}} \\
		\hline
		\textbf{Responsibility:} & \textbf{Collaborators:} \\
		\hline
		Knows user input  & \\
		\hline
		Handles beer prediction output & \\
		\hline
		Knows AccessInfoController & \\
		\hline
		Knows beer information & AccessinfoController\\
		\hline
		
		\end{tabular}
	\end{table}
	
	\begin{table}[ht]
		\centering
		\begin{tabular}{|p{5cm}|p{5cm}|}
		\hline 
		\multicolumn{2}{|l|}{\textbf{Class Name: ForumPage}} \\
		\hline
		\textbf{Responsibility:} & \textbf{Collaborators:} \\
		\hline
		Knows user current location  & \\
		\hline
		Knows AccessInfoController & \\
		\hline
		Knows LCBO and Beerstore locations & NearLocIdentifier\\
		\hline
		Knows beers chosen by experts & AccessInfoController\\
		\hline
	    Displays Map  & \\
	    \hline
		Handles click event for "Facebook", "Twitter", and "Instagram" buttons & Social/MediaShare\\
		\hline
		\end{tabular}
	\end{table}	

	\begin{table}[ht]
		\centering
		\begin{tabular}{|p{5cm}|p{5cm}|}
			\hline 
			\multicolumn{2}{|l|}{\textbf{Class Name: SocialMediaShare}} \\
			\hline
			\textbf{Responsibility:} & \textbf{Collaborators:} \\
			\hline
			Knows user message input  & \\
			\hline
			Checks message word limit & \\
			\hline
			Knows image input & \\
			\hline
			Knows AccessInfoController & \\
			\hline
			Knows media account information & AccessInfoController\\
			\hline 
		\end{tabular}
	\end{table}	
	
	\begin{table}[ht]
		\centering
		\begin{tabular}{|p{5cm}|p{5cm}|}
			\hline 
			\multicolumn{2}{|l|}{\textbf{Class Name: NearLocIdentifier}} \\
			\hline
			\textbf{Responsibility:} & \textbf{Collaborators:} \\
			\hline
			Knows user location & \\
			\hline
			Knows Beer Store and LCBO locations & AccessInfoController\\
			\hline
		\end{tabular}
	\end{table}	
	
	\begin{table}[ht]
		\centering
		\begin{tabular}{|p{5cm}|p{5cm}|}
			\hline 
			\multicolumn{2}{|l|}{\textbf{Class Name: InputCheck}} \\
			\hline
			\textbf{Responsibility:} & \textbf{Collaborators:} \\
			\hline
			Receive input from homepage  & Homepage\\
			\hline
			Receive input from ReSearch & ReSearch\\
			\hline
			Ensure proper input format & InputException\\
			\hline
			Analyze input for keywords & \\
			\hline
			Sort keywords into array & \\
			\hline 
			Return array of keywords & AccessInfoController\\
			\hline 
		\end{tabular}
	\end{table}	

	\begin{table}[ht]
		\centering
		\begin{tabular}{|p{5cm}|p{5cm}|}
			\hline 
			\multicolumn{2}{|l|}{\textbf{Class Name: InputException}} \\
			\hline
			\textbf{Responsibility:} & \textbf{Collaborators:} \\
			\hline
			Receive input error  & InputCheck\\
			\hline
			Display error & \\
			\hline
		\end{tabular}
	\end{table}	
	
	\begin{table}[ht]
		\centering
		\begin{tabular}{|p{5cm}|p{5cm}|}
			\hline 
			\multicolumn{2}{|l|}{\textbf{Class Name: AccessInfoController}} \\
			\hline
			\textbf{Responsibility:} & \textbf{Collaborators:} \\
			\hline
			Access keyword array  & InputCheck\\
			\hline
			Search API information & APIWrapper\\
			\hline
			Request API Updates & UpdateAPI\\
			\hline
			Return keywords to experts & Experts\\
			\hline
			Receive expert information & Experts\\
			\hline
			Reference expert information with API & APIWrapper, Experts\\
			\hline
			Select up to three possible beverage options, placed into an array & \\
			\hline
			Send Array of options to ForumPage & ForumPage \\
			\hline
			
		\end{tabular}
	\end{table}	
	
	\begin{table}[ht]
		\centering
		\begin{tabular}{|p{5cm}|p{5cm}|}
			\hline 
			\multicolumn{2}{|l|}{\textbf{Class Name: ApiWrapper}} \\
			\hline
			\textbf{Responsibility:} & \textbf{Collaborators:} \\
			\hline
			Receive access request  & AccessInfoController\\
			\hline
			Receive update & UpdateAPI\\
			\hline
			Return API request information & AccessInfoController\\
			\hline
		\end{tabular}
	\end{table}	
	
	\begin{table}[ht]
		\centering
		\begin{tabular}{|p{5cm}|p{5cm}|}
			\hline 
			\multicolumn{2}{|l|}{\textbf{Class Name: UpdateWrapper}} \\
			\hline
			\textbf{Responsibility:} & \textbf{Collaborators:} \\
			\hline
			Receive update information  & AccessInfoController\\
			\hline
			Access update information & \\
			\hline
			Sends update & \\
			\hline
		\end{tabular}
	\end{table}	

% End Section

% Division of Labour
%------------------------------------------------------------------------------

\newpage
\clearpage
\appendix
\section{Division of Labour}
\label{sec:division_of_labour}
\begin{table}[!htbp]
\centering
\begin{tabular}{|l|l|l|l|}
\hline
\multicolumn{1}{|c|}{\textbf{Team Member}} & \multicolumn{1}{c|}{\textbf{\begin{tabular}[c]{@{}c@{}}Student \\ Number\end{tabular}}} & \multicolumn{1}{c|}{\textbf{Contribution}} & \multicolumn{1}{c|}{\textbf{Signature}} \\ \hline
Arthur Chen & 1306616 &  &  \\ \hline
Christopher Campbell & 1143732 & &  \\ \hline
Johnny Endrizzi & 1310603 & &  \\ \hline
Mitchell Coovert & 1306701 & &  \\ \hline
Surinder Gill & 1308896 & &  \\ \hline
Terin Dhadda & 1312555 & Title, TOC, Introduction, Use Case Diagram &  \\ \hline
\end{tabular}
\caption{Contributions and Signatures of Team Members}
\end{table}
%------------------------------------------------------------------------------



\end{document}
%------------------------------------------------------------------------------